\documentclass[hyperref={bookmarksopen=false}]{beamer} 

%\usepackage[english]{babel}
\usepackage[ngerman]{babel}
\usepackage{tikz}
%\usepackage[latin1]{inputenc}
\usepackage[utf8]{inputenc}
%fuer externe bilder
\usepackage{graphicx}
\usepackage{listings}

\definecolor{tuRed}{cmyk}{0.1,1.0,0.8,0.0}

\useoutertheme[section]{tubs}

%\setbeamertemplate{itemize items}[ball]
%\setbeamertemplate{itemize items}[square]
\setbeamertemplate{itemize items}[tusquare]

\title[Teamprojekt im Lego-Labor]{TuringBrain IDE}
\subtitle{Ein Turing-Maschinen-Editor und -Simulator aus LEGO} 
\author[]{\tiny Vanessa Baier, Nils Breyer, Phillipp Neumann, Sven Schuster, David Wille}

\institute[TU Braunschweig, ITI]{Technische Universität Braunschweig, ITI}

\date{\today}

\instlogo{ips}
%\titlegraphic{iz}
\titlegraphic{title}

\begin{document}

\frame[plain]{\titlepage} 

\setbeamercolor{frametitle}{fg=white,bg=tu-red}
\frame{
        \frametitle{Inhalt}
        \tableofcontents
        }
\setbeamercolor{frametitle}{fg=black,bg=tu-grey}

\frame{
	\frametitle{Turingmaschine}
	\begin{figure}[!htb]
		\begin{center}
		\begin{tikzpicture}[>=stealth]
		%\draw[help lines,scale=0.5] (0,0) grid (20,5);
		% triangles
		\draw[thick,fill=tuRed] (0,0.75) -- (0.25,1) -- (0.25,0.5) -- (0,0.75);
		\draw[thick,fill=tuRed] (9.75,1) -- (10,0.75) -- (9.75,0.5) -- (9.75,1);
		\draw[thick,fill=tuRed] (5,1.5) -- (5.5,1.5) -- (5.25,1.25) -- (5,1.5);
		% arrows
		\draw[very thick,<-] (4.25,1.5) -- (4.75,1.5);
		\draw[very thick,->] (5.75,1.5) -- (6.25,1.5);
		% tape
		\draw[thick,scale=0.5] (1,1) rectangle +(1,1);
		\draw[thick,scale=0.5] (2,1) rectangle +(1,1);
		\draw[thick,scale=0.5] (3,1) rectangle +(1,1);
		\draw[thick,scale=0.5] (4,1) rectangle +(1,1);
		\draw[thick,scale=0.5] (5,1) rectangle +(1,1);
		\draw[thick,scale=0.5] (6,1) rectangle +(1,1);
		\draw[thick,scale=0.5] (7,1) rectangle +(1,1);
		\draw[thick,scale=0.5] (8,1) rectangle +(1,1);
		\draw[thick,scale=0.5] (9,1) rectangle +(1,1);
		\draw[thick,scale=0.5] (10,1) rectangle +(1,1);
		\draw[thick,scale=0.5] (11,1) rectangle +(1,1);
		\draw[thick,scale=0.5] (12,1) rectangle +(1,1);
		\draw[thick,scale=0.5] (13,1) rectangle +(1,1);
		\draw[thick,scale=0.5] (14,1) rectangle +(1,1);
		\draw[thick,scale=0.5] (15,1) rectangle +(1,1);
		\draw[thick,scale=0.5] (16,1) rectangle +(1,1);
		\draw[thick,scale=0.5] (17,1) rectangle +(1,1);
		\draw[thick,scale=0.5] (18,1) rectangle +(1,1);
		% symbols
		\draw (0.75,0.75) node[draw=none] {\#};
		\draw (1.25,0.75) node[draw=none] {\#};
		\draw (1.75,0.75) node[draw=none] {\#};
		\draw (2.25,0.75) node[draw=none] {\#};
		\draw (2.75,0.75) node[draw=none] {\#};
		\draw (3.25,0.75) node[draw=none] {1};
		\draw (3.75,0.75) node[draw=none] {0};
		\draw (4.25,0.75) node[draw=none] {1};
		\draw (4.75,0.75) node[draw=none] {1};
		\draw (5.25,0.75) node[draw=none] {0};
		\draw (5.75,0.75) node[draw=none] {1};
		\draw (6.25,0.75) node[draw=none] {2};
		\draw (6.75,0.75) node[draw=none] {0};
		\draw (7.25,0.75) node[draw=none] {1};
		\draw (7.75,0.75) node[draw=none] {\#};
		\draw (8.25,0.75) node[draw=none] {\#};
		\draw (8.75,0.75) node[draw=none] {\#};
		\draw (9.25,0.75) node[draw=none] {\#};
		\end{tikzpicture}
		\end{center}
	\end{figure}
  		
}

\section{Hardware / Bauanleitung}

\frame{
	\frametitle{Bedeutung der Bit Kombinationen}
  	\begin{figure}[!htb]
		\begin{center}
		\begin{tikzpicture}
		%\draw[help lines,scale=0.5] (0,0) grid (20,5);
		\draw[thick,scale=0.5] (1,1) rectangle +(1,1);
		\draw[thick,scale=0.5] (3,1) rectangle +(1,1);
		\draw[thick,scale=0.5,fill=tuRed] (1,3) rectangle +(1,1);
		\draw[thick,scale=0.5,fill=tuRed] (3,3) rectangle +(1,1);
		\draw (1.25,0.25) node[draw=none] {\#};

		\draw[thick,scale=0.5] (6,1) rectangle +(1,1);
		\draw[thick,scale=0.5,fill=tuRed] (8,1) rectangle +(1,1);
		\draw[thick,scale=0.5,fill=tuRed] (6,3) rectangle +(1,1);
		\draw[thick,scale=0.5] (8,3) rectangle +(1,1);
		\draw (3.75,0.25) node[draw=none] {0};

		\draw[thick,scale=0.5,fill=tuRed] (11,1) rectangle +(1,1);
		\draw[thick,scale=0.5] (13,1) rectangle +(1,1);
		\draw[thick,scale=0.5] (11,3) rectangle +(1,1);
		\draw[thick,scale=0.5,fill=tuRed] (13,3) rectangle +(1,1);
		\draw (6.25,0.25) node[draw=none] {1};

		\draw[thick,scale=0.5,fill=tuRed] (16,1) rectangle +(1,1);
		\draw[thick,scale=0.5,fill=tuRed] (18,1) rectangle +(1,1);
		\draw[thick,scale=0.5] (16,3) rectangle +(1,1);
		\draw[thick,scale=0.5] (18,3) rectangle +(1,1);
		\draw (8.75,0.25) node[draw=none] {2};
		\end{tikzpicture}
		\end{center}
	\end{figure}
}

\frame{
	\frametitle{Live Demo}
	TM erstellen + Sim  		
}

\frame{
	\frametitle{Features}
	Sim ohne Delay\\
	Export  		
}

\frame{
	\frametitle{Optional}
	Brainfuck\\
	2-Band-Sim (am Ende)\\
	Doku		
}

\frame{
  \begin{center}\huge
    Fragen?
  \end{center}

}

\end{document}   
