\documentclass[hyperref={bookmarksopen=false}]{beamer} 

%\usepackage[english]{babel}
\usepackage[ngerman]{babel}
\usepackage{pgf,pgfarrows,pgfnodes,pgfautomata,pgfheaps,pgfshade}
%\usepackage[latin1]{inputenc}
\usepackage[utf8]{inputenc}
%fuer externe bilder
\usepackage{graphicx}
\usepackage{listings}


\useoutertheme[section]{tubs}

%\setbeamertemplate{itemize items}[ball]
%\setbeamertemplate{itemize items}[square]
\setbeamertemplate{itemize items}[tusquare]

\title[Grundlagen der Sicherheit in Netzen und verteilten
Systemen]{Der Geburtstagsangriff auf Hash-Funktionen}
\subtitle{} 
\author[]{David Wille, Nils Breyer}

\institute[TU Braunschweig, ITI]{Technische Universität Braunschweig, ITI}

\date{\today}

\instlogo{ips}
%\titlegraphic{iz}
\titlegraphic{torte}



\begin{document}

\frame[plain]{\titlepage} 

\setbeamercolor{frametitle}{fg=white,bg=tu-red}
\frame{
        \frametitle{Inhalt}
        \tableofcontents
        }
\setbeamercolor{frametitle}{fg=black,bg=tu-grey}

\section{Das Geburtstagsparadoxon}

\frame{
	\frametitle{Ursprung}
  		\begin{itemize}
  			\item österreichischer Mathematiker Richard von Mises
  			\pause
  			\item Wie hoch ist die Wahrscheinlichkeit, dass bei 23 Personen mindestens zwei am selben Tag Geburtstag haben?
  			\pause
  			\item Erstaunlich ist: $P > 50\%$
  			\pause
  			\item bei 50 Personen: $P > 97\%$
  		\end{itemize}
}


\frame{
  \begin{center}\huge
    Fragen?
  \end{center}

}

\end{document}   
