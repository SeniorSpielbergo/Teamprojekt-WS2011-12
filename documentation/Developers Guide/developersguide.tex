\documentclass[%
  a4paper,%
  11pt,% <10pt, 9pt>
  %style=screen,
  %sender=bottom,
  blue,% <orange, green, violet>
  %rgb, <cmyk>
  %mono,
  hyperref	% tubs color for hyperref
  ]{tubsartcl}
\usepackage[utf8]{inputenc}
\usepackage{hyperref}
\usepackage{tikz}
%\usepackage[a4paper,margin=1cm,landscape]{geometry}
\usetikzlibrary{positioning,shadows,arrows}

\setlength{\parindent}{0cm}
 
% Titelseiten-Elemente
\title{TuringBrain IDE \LARGE 1.0}
\subtitle{Developer's Guide}
\author{\small Vanessa Baier, Nils Breyer, Phillipp Neumann,\\ Sven Schuster, David Wille}
\logo{\includegraphics{ips}}
\titleabstract{TuringBrain IDE -- Developer's Guide}
\titlepicture{title}
% Rückseiten-Elemente
\address{
  Advisor: Matthias Hagner\\\\
  Technische Universität Braunschweig\\
  Institut für Programmierung und Reaktive Systeme\\
  Mühlenpfordtstr. 23\\
  38106 Braunschweig}
\backpageinfo{
}

\begin{document}

\maketitle[image,logo=right]%[<plain/image/imagetext>,<logo=left/right>]

\tableofcontents
\newpage
\section{Introduction}

\section{Code -- LEGO\textregistered\, tape}

In the nxt Project you have two main classes, MainMaster and MainSlave. These two classes are responsible for receiving commands the Master and Slave robot reply with a \'.\'. The Master also have methods to send the read character to the pc receiver. Functions for the tape like moving left and right but it also implements a counter and a timer. \\
Both Master and Slave Main classes import the common class that has some functions that both classes use.\\
There are some more common classes that can be used for testing the sensors or calibrate them.
%\subsection{Master NXT module}

%\subsubsection{MainMaster}

%\subsection{Slave NXT module}

%\subsubsection{MainSlave}

%\subsubsection{TetrisSound}


%\subsection{Common NXT classes}

%\subsubsection{Calibrate}

%\subsubsection{Common}

%\subsubsection{Tape}

%\subsubsection{Timer}

%\subsubsection{SensorTest}

\section{PC Packages}

\begin{center}
\begin{tikzpicture}[
    fact/.style={rectangle, draw=none, rounded corners=1mm, fill=blue, drop shadow,
        text centered, anchor=north, text=white},
    state/.style={circle, draw=none, fill=tuRed,
        text centered, anchor=north, text=white},
    leaf/.style={circle, draw=none, fill=red, circular drop shadow,
        text centered, anchor=north, text=white},
    level distance=0.5cm, growth parent anchor=south
]
\node (State00) [state] {PC Project} [->] 
	[sibling distance=6cm]
	child{[sibling distance=2cm]
		node (State01) [state] {gui}
		child{
			node (State02) [state] {brainfuck}
		}
		child{
			node (State03) [state] {turing}			
		}
	}
	child{[sibling distance=2cm]
		node (State05) [state] {machine}
		child{
			node (State06) [state] {turing}
		}
		child{
			node (State07) [state] {brainfuck}
		}		
	}
	child{[sibling distance=2cm]
		node (State09) [state] {tape}
	}
;
\end{tikzpicture}
\end{center}

\section{Code -- GUI}


\subsection{Package gui}

\subsubsection{AboutDialog}

\subsubsection{AppData}

\subsubsection{CustomTable}

\subsubsection{Editor}

\subsubsection{ErrorDialog}

\subsubsection{MachineEditor}

\subsubsection{OrganizeRobots}

\subsubsection{PresimulationDialogue}

\subsubsection{RunWindow}

\subsubsection{SimulationWindow}

\subsubsection{WelcomeScreen}

\subsubsection{WelcomeScreenGroup}

\subsubsection{WelcomeScreenLine}

\subsection{Package gui.brainfuck}

\subsubsection{BrainfuckEditor}


\subsection{Package gui.turing}

\subsubsection{NewTMDialogue}

\subsubsection{Palette}

\subsubsection{PropertiesEdge}

\subsubsection{PropertiesEdgeEdit}

\subsubsection{PropertiesState}

\subsubsection{PropertiesTextbox}

\subsubsection{PropertiesTuringMachine}

\subsubsection{ToolBox}

\subsubsection{TuringMachineEditor}


\subsection{Package machine}

\subsubsection{ExtensionFileFilter}

\subsubsection{Machine}

\subsubsection{Simulation}


\subsubsection{Package machine.brainfuck}

\subsubsection{BrainfuckMachine}

\subsubsection{BrainfuckSimulation}


\subsection{Package machine.turing}

\subsubsection{Edge}

\subsubsection{Frame}

\subsubsection{InOut}

\subsubsection{Point}

\subsubsection{State}

\subsubsection{Textbox}

\subsubsection{Transition}

\subsubsection{TuringMachine}

\subsubsection{TuringSimulation}


\subsection{Package tape}

\subsubsection{ConsoleTape}

\subsubsection{DisplayableTape}

\subsubsection{GraphicTape}

\subsubsection{GraphicTapePanel}

\subsubsection{LEGOTape}

\subsubsection{MasterRobot}

\subsubsection{Robot}

\subsubsection{SlaveRobot}

\subsubsection{Tape}

\subsubsection{TapeException}


\makebackpage[trisec]%[<plain/info/addressinfo>]

\end{document}