\documentclass[%
  a4paper,%
  11pt,% <10pt, 9pt>
  %style=screen,
  %sender=bottom,
  blue,% <orange, green, violet>
  %rgb, <cmyk>
  %mono,
  hyperref	% tubs color for hyperref
  ]{tubsartcl}
\usepackage[utf8]{inputenc}
\usepackage{hyperref}
\usepackage{tikz}
\usepackage{textcomp}
%\usepackage[a4paper,margin=1cm,landscape]{geometry}
\usetikzlibrary{positioning,shadows,arrows}

\setlength{\parindent}{0cm}
 
% Titelseiten-Elemente
\title{TuringBrain IDE \LARGE 1.0}
\subtitle{Developer's Guide}
\author{\small Vanessa Baier, Nils Breyer, Phillipp Neumann,\\ Sven Schuster, David Wille}
\logo{\includegraphics{ips}}
\titleabstract{TuringBrain IDE -- Developer's Guide}
\titlepicture{title}
% Rückseiten-Elemente
\address{
  Advisor: Matthias Hagner\\\\
  Technische Universität Braunschweig\\
  Institut für Programmierung und Reaktive Systeme\\
  Mühlenpfordtstr. 23\\
  38106 Braunschweig}
\backpageinfo{
}

\begin{document}

\maketitle[image,logo=right]%[<plain/image/imagetext>,<logo=left/right>]

\tableofcontents
\newpage
\section{Introduction}

The developers guide gives you a short introduction, if you want to understand or extend the TuringBrain IDE source code. The guide starts with a short overview over the code for the NXT and PC parts. Then we provide an overview over the protocol between the PC and the NXT modules. Additionially this guide contains some tips and instructions for common extension scenarios, like the introduction of new machine types.

This guide is not a complete documentation of the code, but its purpose is to give you a short overview. The detailled documentation can be found directly in the code as javadoc. Use javadoc to generate the HTML code documentation for each project (nxt and pc).

\section{PC code}

The code of the pc project can be found in the src/pc folder. The code is structured in packages as shown in figure \ref{fig:pc-packages}.
  \begin{figure}[h]
    \centering
    \begin{tikzpicture}[ fact/.style={rectangle, draw=none, rounded
        corners=1mm, fill=blue, drop shadow, text centered,
        anchor=north, text=white}, state/.style={circle, draw=none,
        fill=tuRed, text centered, anchor=north, text=white, minimum
        width=2.0cm}, leaf/.style={circle, draw=none, fill=red,
        circular drop shadow, text centered, anchor=north,
        text=white}, level distance=0.5cm, growth parent anchor=south,
      >=stealth ]
      \node (State00) [state] {PC Project} [very thick,->] [sibling
      distance=6cm] child{[sibling distance=3cm] node (State01)
        [state] {gui} child{ node (State02) [state] {brainfuck} }
        child{ node (State03) [state] {turing} } } child{[sibling
        distance=3cm] node (State05) [state] {machine} child{ node
          (State06) [state] {turing} } child{ node (State07) [state]
          {brainfuck} } } child{[sibling distance=2cm] node (State09)
        [state] {tape} } ;
    \end{tikzpicture}
    
    \caption{The pc project package structure}
\label{fig:pc-packages}
\end{figure}

All UI classes are placed in the gui package. The machine package contains the classes for machines and their simulation and the tape packages contains the classes used for the graphical and LEGO\textregistered\,tapes. For detailled documentation of the code generate the javadoc HTML files from the code.

\subsection{AppData}

The tape styles, like the example files are installed in the user's home directory, when you start the program for the first time. If you change them, you wil have to change the version number in the pc/gui/AppData.java file, so that the files are being updated in the user directory.


The folder where the AppData is stored differs with your operating system:

\begin{itemize}
\item Linux: \$HOME/.TuringBrain IDE
\item Mac OS X: \$HOME/Library/Application Support/Turing Brain IDE
\item Other: \$HOME/Turing Brain IDE
\end{itemize}

\subsection{Adding new machine types}

If you want to add a new machine type (examples could be new automata types, Load-Store-Architectures, Assembler-Code, esoteric programming languages like COW etc.) you have to create new packages for the machine (machine.yourmachinetype) and the editor (gui.yourmachinetype). The machine package has to contain a class that inherits from \emph{machine.Machine} (contains methods for loading and saving a machine) and \emph{machine.Simulation} (contains methods to simulate a machine). The editor package must contain an inheritance of \emph{gui.MachineEditor} (contains a panel that allows you to edit the machine). 
Additionally, to be able to open your machine files, you have to adapt the FileChooser in gui.Editor.openFile() and add your machine's file extension to it. \\
You can also add your machine type to the menu item \textit{New} in the menubar of gui.Editor. Optionally you can add your machine type as a new tab to \emph{gui.WelcomeScreen}.\\
For more information on the methods that have to be implemented you can view our Javadoc. As a reference implementation you can look at the Brainfuck packages, which are kind of a minimal implementation of a machine, which are much more handy than the corresponding Turing machine packages.

\subsection{Adding new tape types}

Adding new tape types is a less common task, but examples could be that you want to program a tape that works over LAN on another computer, a tape that can be moved by the user manually and not only show the content near the head, or a tape that directly writes its content to a file.

For this task you have to either inherit from tape.Tape (if you don't want that your tape shows up in the simulation window) or from tape.DisplayableTape (if you want that it shows up in the simulation window). To be able to select the new tape type for the simulation you need to modify the relevant methods in the gui.RunWindow class and add your tape type to the Type enum in the tape.Tape class.

\subsection{Adding new tape styles}
\label{sec:adding-new-tape}

The tape panels (showed in the simulation window) are themeable. The themes are stored in the source code in the pc/tape/images/styles directory. Each folder in this directory is a theme (the folder name is the theme name). Every tape style folder must contain two images: head.png and tape\_field.png. These images should have exactly the same size. Recommend is a size of 28 x 32, but you can also use diffrent sizes. The font size is scaled dynamically with the image size. The head.png should be transparent in the middle, otherwise you won't see the field content.


\section{LEGO\textregistered\, tape code}
\label{sec:lego}

The project nxt contains classes to operate on the LEGO\textregistered\, modules. There are the two main classes, MainMaster and MainSlave which are responsible for receiving and executing commands. The Master also has methods to send the read character to the pc receiver. Functions for the tape like moving left and right are provided by the class Tape. It also implements a counter to determine the position of the tape and a security timer. \\
Also the Common class is implemented which is used by both Master and Slave providing functions like pushing bits, playing sounds etc.\\
There are some more common classes that can be used for testing the sensors or calibrate them.

\section{Network protocol}

This section is an overview over the protocol that is used for the communication between the PC and the LEGO\textregistered\, NXT robots. The communication works via bluetooth and is character based. Usually one character is one command. Some commands (i.e. machine name command) are followed by a character sequence of arbitrary length terminated by a newline character.

The following tables show the complete list of commands sent by the pc to the master. The NXT modules only respond to previous commands by the pc, they will never send commands spontaneously to the pc. The response to most commands is either a ``.'' (ok, success) or ``!'' (failure). The read command is responded by the read symbol.

\begin{minipage}{5cm}
\subsection{Master}
\begin{tabular}{c|l}
command & description\\
\hline
q & quit\\
t<name>\textbackslash n & name of the tape\\
r & read\\
w & write\\
L & move left\\
R & move right
\end{tabular}
\vspace{1cm}
\end{minipage}
\hspace{2cm}
\begin{minipage}{5cm}
\subsection{Slave}
\begin{tabular}{c|l}
command & description\\
\hline
q & quit\\
w & write\\
S & starting sound\\
s & set sound off\\
m & mute sound\\
M & unmute sound\\
n<name>\textbackslash n & machine name\\
x<name>\textbackslash n & state name
\end{tabular}
\end{minipage}

\makebackpage[trisec]%[<plain/info/addressinfo>]

\end{document}