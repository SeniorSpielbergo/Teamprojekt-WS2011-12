\documentclass[%
  a4paper,%
  11pt,% <10pt, 9pt>
  %style=screen,
  %sender=bottom,
  blue,% <orange, green, violet>
  %rgb, <cmyk>
  %mono,
  hyperref	% tubs color for hyperref
  ]{tubsartcl}
\usepackage[utf8]{inputenc}
\usepackage{hyperref}
\usepackage{tikz}
\usepackage{textcomp}
%\usepackage[a4paper,margin=1cm,landscape]{geometry}
\usetikzlibrary{positioning,shadows,arrows}

\setlength{\parindent}{0cm}
 
% Titelseiten-Elemente
\title{TuringBrain IDE \LARGE 1.0}
\subtitle{Developer's Guide}
\author{\small Vanessa Baier, Nils Breyer, Phillipp Neumann,\\ Sven Schuster, David Wille}
\logo{\includegraphics{ips}}
\titleabstract{TuringBrain IDE -- Developer's Guide}
\titlepicture{title}
% Rückseiten-Elemente
\address{
  Advisor: Matthias Hagner\\\\
  Technische Universität Braunschweig\\
  Institut für Programmierung und Reaktive Systeme\\
  Mühlenpfordtstr. 23\\
  38106 Braunschweig}
\backpageinfo{
}

\begin{document}

\maketitle[image,logo=right]%[<plain/image/imagetext>,<logo=left/right>]

\tableofcontents
\newpage
\section{Introduction}

\section{Code -- LEGO\textregistered\, tape}

The project nxt contains classes to operate on the LEGO\textregistered\, modules. There are the two main classes, MainMaster and MainSlave which are responsible for receiving and executing commands. The Master also has methods to send the read character to the pc receiver. Functions for the tape like moving left and right are provided by the class Tape. It also implements a counter to determine the position of the tape and a security timer. \\
Also the Common class is implemented which is used by both Master and Slave providing functions like pushing bits, playing sounds etc.\\
There are some more common classes that can be used for testing the sensors or calibrate them.

\section{PC Packages}

\begin{center}
\begin{tikzpicture}[
    fact/.style={rectangle, draw=none, rounded corners=1mm, fill=blue, drop shadow,
        text centered, anchor=north, text=white},
    state/.style={circle, draw=none, fill=tuRed,
        text centered, anchor=north, text=white, minimum width=2.0cm},
    leaf/.style={circle, draw=none, fill=red, circular drop shadow,
        text centered, anchor=north, text=white},
    level distance=0.5cm, growth parent anchor=south, >=stealth
]
\node (State00) [state] {PC Project} [very thick,->] 
	[sibling distance=6cm]
	child{[sibling distance=3cm]
		node (State01) [state] {gui}
		child{
			node (State02) [state] {brainfuck}
		}
		child{
			node (State03) [state] {turing}			
		}
	}
	child{[sibling distance=3cm]
		node (State05) [state] {machine}
		child{
			node (State06) [state] {turing}
		}
		child{
			node (State07) [state] {brainfuck}
		}		
	}
	child{[sibling distance=2cm]
		node (State09) [state] {tape}
	}
;
\end{tikzpicture}
\end{center}



\section{Protocol}

This section is an overview over the protocol that is used for the communication between the PC and the LEGO\textregistered\, NXT robots. The communication works via bluetooth and is character based. Usually one character is one command. Some commands (i.e. machine name command) are followed by a character sequence of arbitrary length terminated by a newline character.

The following tables show the complete list of commands sent by the pc to the master. The NXT modules only respond to previous commands by the pc, they will never send commands spontaneously to the pc. The response to most commands is either a ``.'' (ok, success) or ``!'' (failure). The read command is responded by the read symbol.

\begin{minipage}{5cm}
\subsection{Master}
\begin{tabular}{c|l}
command & description\\
\hline
q & quit\\
t<name>\textbackslash n & name of the tape\\
r & read\\
w & write\\
L & move left\\
R & move right
\end{tabular}
\vspace{1cm}
\end{minipage}
\hspace{2cm}
\begin{minipage}{5cm}
\subsection{Slave}
\begin{tabular}{c|l}
command & description\\
\hline
q & quit\\
w & write\\
S & starting sound\\
s & set sound off\\
m & mute sound\\
M & unmute sound\\
n<name>\textbackslash n & machine name\\
x<name>\textbackslash n & state name
\end{tabular}
\end{minipage}


\makebackpage[trisec]%[<plain/info/addressinfo>]

\end{document}