\documentclass[%
  a4paper,%
  11pt,% <10pt, 9pt>
  style=screen,
  %sender=bottom,
  blue,% <orange, green, violet>
  %rgb, <cmyk>
  %mono
  ]{tubsartcl}
\usepackage[utf8x]{inputenc}
 
% Titelseiten-Elemente
\title{TuringBrain IDE \LARGE 1.0}
\subtitle{User Guide}
\author{\small Vanessa Baier, Nils Breyer, Phillipp Neumann,\\ Sven Schuster, David Wille}
\logo{\includegraphics{ips}}
\titleabstract{TuringBrain IDE -- User Guide}
\titlepicture{title}
% Rückseiten-Elemente
\address{
  Advisor: Matthias Hagner\\\\
  Technische Universität Braunschweig\\
  Institut für Programmierung und Reaktive Systeme\\
  Mühlenpfordtstr. 23\\
  38106 Braunschweig}
\backpageinfo{
}

\begin{document}

\maketitle[image,logo=right]%[<plain/image/imagetext>,<logo=left/right>]
\makebackpage[trisec]%[<plain/info/addressinfo>]

\tableofcontents
\newpage
\section{Introduction}

Test \textcolor{tuSecondary}{Dies ist ein Text in \texttt{tuSecondary}.}\bigskip


\begin{itemize}
  \item Aufzählungspunkt Eins
  \item Aufzählungspunkt Zwei
    \begin{itemize}
      \item Unter-Aufzählungspunkt Eins
      \item Unter-Aufzählungspunkt Zwei
    \end{itemize}
  \item Aufzählungspunkt Drei
\end{itemize}

\section{Getting started}


\section{Turing Machine editor}


\section{Brainfuck editor}


\section{Simulation}

\section{Hardware -- LEGO\textregistered\, tape}


\end{document}
