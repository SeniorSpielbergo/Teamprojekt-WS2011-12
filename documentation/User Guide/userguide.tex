\documentclass[%
  a4paper,%
  11pt,% <10pt, 9pt>
  %style=screen,
  %sender=bottom,
  blue,% <orange, green, violet>
  %rgb, <cmyk>
  %mono,
  hyperref	% tubs color for hyperref
  ]{tubsartcl}
\usepackage[utf8]{inputenc}
\usepackage{hyperref}
 
% Titelseiten-Elemente
\title{TuringBrain IDE \LARGE 1.0}
\subtitle{User Guide}
\author{\small Vanessa Baier, Nils Breyer, Phillipp Neumann,\\ Sven Schuster, David Wille}
\logo{\includegraphics{ips}}
\titleabstract{TuringBrain IDE -- User Guide}
\titlepicture{title}
% Rückseiten-Elemente
\address{
  Advisor: Matthias Hagner\\\\
  Technische Universität Braunschweig\\
  Institut für Programmierung und Reaktive Systeme\\
  Mühlenpfordtstr. 23\\
  38106 Braunschweig}
\backpageinfo{
}

\begin{document}

\maketitle[image,logo=right]%[<plain/image/imagetext>,<logo=left/right>]

\tableofcontents
\newpage
\section{Introduction}

TuringBrain IDE allows you to develop and debug your own TuringMachines in an easy to use WYSIWYG (what you see is what you get) environment. It also enables you to program and run Brainfuck code.



\subsection{Features}

\begin{itemize}
  \item Edit Turing Machines by graphically editing the state graph
  \item Create Turing Machines with multiple tapes
  \item Write Brainfuck programs within the integrated code editor
  \item Simulate your machines and programs using the integrated simulation
  \item Simulate on special LEGO-Tape (hardware needed), graphically animated on the screen or on the console
  \item Pause simulation, Debug step-by-step
  \item Live-highlighting of the current state and edge during the simulation
  \item Save machines as .tm (an XML-Format) respective .bf
  \item Copy \& Paste, Undo \& Redo (not yet for Turing Machines)
  \item Export to Latex, SVG, and PNG
  \item and many more features...
\end{itemize}

\section{Getting started}

\subsection{Installation}
\label{sec:installation}

\subsection{Welcome page}
\label{sec:welcome-page}


\section{Turing Machine editor}

\subsection{Edit machine properties}
\label{sec:edit-mach-prop}

\subsection{Adding and changing states}
\label{sec:add-edit-states}

\subsection{Adding and changing edges}
\label{sec:adding-chang-edges}

\subsection{Edit transitions}
\label{sec:edit-transitions}

\subsection{Via points}
\label{sec:via-points}

\subsection{Frames and textboxes}
\label{sec:frames-textboxes}

\subsection{Export}
\label{sec:export}

\section{Brainfuck editor}



\section{Simulation}



\section{Hardware -- LEGO\textregistered\, tape}

\begin{figure}[!htb]
\begin{center}
\includegraphics[scale=0.4]{graphics/topview.jpg}
\end{center}
\caption{Overview of the main machine}
\end{figure}

\begin{figure}[!htb]
\begin{center}
\includegraphics[scale=0.4]{graphics/1.jpg}
\end{center}
\caption{Image taken from position 1}
\end{figure}

\begin{figure}[!htb]
\begin{center}
\includegraphics[scale=0.4]{graphics/2.jpg}
\end{center}
\caption{Image taken from position 2}
\end{figure}

\begin{figure}[!htb]
\begin{center}
\includegraphics[scale=0.4]{graphics/3.jpg}
\end{center}
\caption{Image taken from position 3}
\end{figure}

\begin{figure}[!htb]
\begin{center}
\includegraphics[scale=0.3]{graphics/4.jpg}
\end{center}
\caption{Image taken from position 4}
\end{figure}

\begin{figure}[!htb]
\begin{center}
\includegraphics[scale=0.4]{graphics/bridge1.jpg}
\end{center}
\caption{Position of the bridge}
\end{figure}

\begin{figure}[!htb]
\begin{center}
\includegraphics[scale=0.4]{graphics/bridge2.jpg}
\end{center}
\caption{Position of the bridge}
\end{figure}

\begin{figure}[!htb]
\begin{center}
\includegraphics[scale=0.3]{graphics/tape.jpg}
\end{center}
\caption{Tape}
\end{figure}

\begin{figure}[!htb]
\begin{center}
\includegraphics[scale=0.3]{graphics/sensors.jpg}
\end{center}
\caption{Position of the sensors}
\end{figure}

\begin{figure}[!htb]
\begin{center}
\includegraphics[scale=0.4]{graphics/topview_pushers.jpg}
\end{center}
\caption{Position of the pushers}
\end{figure}

\begin{figure}[!htb]
\begin{center}
\includegraphics[scale=0.3]{graphics/pusher.jpg}
\end{center}
\caption{Pusher}
\end{figure}

\makebackpage[trisec]%[<plain/info/addressinfo>]

\end{document}
