\documentclass[a4paper,12pt]{article}

\usepackage{ngerman}
\usepackage[utf8]{inputenc}
\usepackage{t1enc}

\begin{document}

\author{Vanessa Baier, Nils Breyer, Philipp Neumann, Sven Schuster, David Wille}
\title{Teamprojekt mit LEGO-Robotern am IPS im \mbox{WS 2011/12} - Exposé}

\maketitle

\section{Grundidee}

Das Ziel unseres Teamprojektes ist es, dass Konzept der Turingmaschine (TM) aus der theoretischen Informatik mit LEGO-Robotern zu simulieren und zu veranschaulichen. Da die TM ein theoretisches Konzept ist, wird es bei der praktischen Umsetzung zwangsläufig einige Einschränkungen geben.

\subsection{Turingmaschine in der theoretischen Informatik}

In der theoretischen Informatik versteht besteht eine TM aus einem unendlich langen Speicherband mit unendlich vielen Feldern. In jedem dieser Felder ist zu jedem Zeitpunkt genau ein Zeichen aus einem definierten endlichen Alphabet gespeichert. Ein spezielles Symbol, das sogenannte "`Blanksymbol"' dient als Repräsentant für ein leeres Feld.
 
Weiterhin gibt es einem Lese- und Schreibkopf, der sich auf dem Speicherband feldweise bewegen und die Zeichen verändern kann. Dieser Kopf wird dabei durch ein entsprechendes Programm (formal Übergangsfunktionen) gesteuert.

\subsection{Turingmaschine mit LEGO-Robotern}

Natürlich lässt sich eine wie oben beschriebene Maschine nicht mit LEGO realisieren, da dies unendliche Ressourcen benötigen würde. Wir planen daher eine Simulation zu realisieren, die in etwa wie folgt funktioniert:




\section{Pflichtziele}

\section{Wunschziele}

\section{Ideen zur Umsetzung}

\section{Zeitplan}



\end{document}
