\documentclass[a4paper,12pt]{article}

\usepackage{ngerman}
\usepackage[utf8]{inputenc}
\usepackage{t1enc}

\setlength{\parindent}{0pt}

\begin{document}

\newcommand{\TODO}[1]{{\bf{TODO: #1}}}

\author{Vanessa Baier, Nils Breyer, Philipp Neumann, Sven Schuster, David Wille}
\title{Teamprojekt mit LEGO-Robotern am IPS im \mbox{WS 2011/12} - Exposé}

\maketitle

\section{Grundidee}

Das Ziel unseres Teamprojektes ist es, dass Konzept der Turingmaschine (TM) aus der theoretischen Informatik mit LEGO-Robotern zu simulieren und zu veranschaulichen. Da die TM ein theoretisches Konzept ist, wird es bei der praktischen Umsetzung zwangsläufig einige Einschränkungen geben.

\subsection{Turingmaschine in der theoretischen Informatik}

In der theoretischen Informatik besteht eine TM aus einem unendlich langen Speicherband mit unendlich vielen Feldern. In jedem dieser Felder ist zu jedem Zeitpunkt genau ein Zeichen aus einem definierten endlichen Alphabet gespeichert. Ein spezielles Symbol, das sogenannte "`Blanksymbol"', dient als Repräsentant für ein leeres Feld.
 
Weiterhin gibt es einem Schreib- und Lesekopf, der sich auf dem Speicherband feldweise bewegen und die Zeichen verändern kann. Dieser Kopf wird dabei durch ein entsprechendes Programm (formal Übergangsfunktion) gesteuert.

\subsection{Simulation mit LEGO-Robotern}

Natürlich lässt sich eine wie oben beschriebene Maschine nicht mit LEGO realisieren, da dies unendliche Ressourcen benötigen würde. Wir planen daher eine Simulation zu realisieren, die wie folgt aufgebaut ist:

Es soll ein Programm geben, mit dem man die Maschine am Rechner programmieren kann. D.h. die Zustandsübergänge und Aktionen, die der Schreib- und Lesekopf beim Übergang ausführen soll, werden vorher am Rechner festgelegt. Ein erstelltes Programm kann dann auf die eigentliche Maschine übertragen und dort ausgeführt werden.

Die Maschine selbst könnte idealerweise als Mehrbandmaschine (also z.B. drei unabhängige Bänder und Schreib-/Leseköpfe) ausgeführt werden. Ein zusätzlicher Roboter könnte als Zentraleinheit fungieren und die Programmlogik ausführen und entsprechende Anweisungen an die Schreib-/Leseköpfe senden. Gleichzeitig könnte er über das Display den aktuellen Zustand anzeigen, in dem sich die TM befindet.

Die Bänder selber werden in der Anzahl der Felder natürlich beschränkt sein, womit nicht jedes Programm simuliert werden kann. Die Symbole können durch unterschiedliche Farben codiert werden. Ein Eingabewort, dass vor der Ausführung eines Programms auf dem Band steht, kann so über Farbsensoren am Schreib-/Lesekopf erkannt werden.

\TODO{so ok? eventuell noch etwas mehr zur umsetzung}

\section{Pflichtziele}

\begin{itemize}
\item Es wird mit LEGO-Robotern eine mechanische Simulation einer Turingmaschine mit mindestens einem Band und einem Schreib-/Lesekopf gebaut.
\item Die Bänder enthalten mindestens 10 Felder.
\item Auf jeder Bandzelle kann eins von mindestens 3 verschiedenen Symbolen (davon ein Blanksymbol) gespeichert werden. Die Symbole auf dem Band können z.B. durch Bauteile aus verschiedenen Farben realisiert werden.
\item Es ist möglich vor der Ausführung eines Programms ein Eingabewort auf das Band zu schreiben, dass die Maschine dann lesen kann. 
\item Es gibt ein Program mit dem auf einem Rechner ein Programm für die Turingmaschine erstellt und dann ausgeführt werden kann.
\item \TODO{fehlt noch etwas wichtiges?}
\end{itemize}

\section{Wunschziele}

\begin{itemize}
\item Auf einem der LEGO-Roboter wird angezeigt, in welchem Zustand sich die Maschine gerade befindet.
\item Es werden zwei oder drei Bänder mit unabhängigen Schreib-/Leseköpfen realisiert.
\item Es sollen möglichst mehr als 3 Symbole und mehr als 10 Felder realisiert werden.
\item Die Übertragung des Programms auf die Maschine erfolgt kabellos (über Bluetooth).
\item Die Programmierung am Rechner könnte graphisch (als Zustandsgraph mit Übergängen) umgesetzt werden, um die Eingabe zu erleichtern.
\item \TODO{weitere ideen?}
\end{itemize}

\section{Ideen zur Umsetzung}

\TODO{hier könnte man ein bisschen darauf eingeben, wie die hardware-umsetzung mit LEGO aussehen könnte...}

\section{Zeitplan}

Der folgende Zeitplan soll nur als grobe Orientierung dienen.\\

\begin{center}
  \begin{tabular}{llll}
    \textbf{KW} & \textbf{von} & \textbf{bis} & \textbf{Phase}\\
    \hline
    ...42 &  & 23.10. & Vorplanung\\ 
    \hline
    43 & 24.10. & 30.10. & \\
    \hline
    44 & 31.10. & 06.11. & \\
    \hline
    45 & 07.11. & 13.11. & \\
    \hline
    46 & 14.11. & 20.11. & \\
    \hline
    47 & 21.11. & 27.11. & \\
    \hline
    48 & 28.11. & 04.12. & \\ 
    \hline
    49 & 05.12. & 11.12. & \\
    \hline
    50 & 12.12. & 18.12. & \\
    \hline
    51 & 19.12. & 25.12. & \emph{ab 24.12. Weihnachtsferien}\\
    \hline
    52 & 26.12. & 01.01. & \emph{Weihnachtsferien}\\
    \hline
    01 & 02.01. & 08.01. & \emph{Weihnachtsferien}\\
    \hline
    02 & 09.01. & 15.01. & \\
    \hline
    03 & 16.01. & 22.01. & \\
    \hline
    04 & 23.01. & 29.01. & \\
    \hline
    05 & 30.01. & 05.02. & Abgabe und Bewertung\\
    \hline
    06 & 06.02. & 12.02. & Bewertung\\
    \hline
  \end{tabular}
\end{center}


\TODO{phasen/meilensteine in den zeitplan einfügen}

\end{document}
